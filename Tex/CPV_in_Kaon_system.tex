\section{$\hat{C}\hat{P}$V in Kaon System} 

\subsection{Neutral Kaon Mixing}

As mentioned $\hat{C}\hat{P}$V was first observed in the neutral kaon system. Direct and indirect $\hat{C}\hat{P}$V have been observed but it is found that the process is entirely dominated by the indirect method [Zeng, need better reference]. Essential to these mechanisms is the mixing between the neutral Kaon and its anti-particle, corresponding to the states $\ket{K^{0}}$ and $\ket{\bar{K}^{0}}$. These have quark compositions of $d \bar{s}$ and $s \bar{d}$, respectively. 

In interactions involving the strong or EM force, the quantum number strangeness, which tells us the number of strange quarks in a particle, must be conserved. For the weak force it is found that, like parity, this symmetry is not conserved. Due to this many processes forbidden for the strong and EM interactions are allowed through the weak force. This violation is what makes mixing possible. Mixing is the decay of a particle into its anti-particle and can only take place when a particle is its own anti-particle, or if the particles differ by a quantum number which is not conserved by some interaction. This is the case in neutral Kaon mixing, also know as Kaon oscillations. The neutral Kaon and its anti-particle have opposite strangeness but can decay into each other through the strangeness violating weak force. See Fig.(add in feynman digram of M+S pg 289) 

Analogous to the mixing of mass eigenstate quarks to different quark flavours, it is found that the neutral Kaon states observed in nature do not correspond to eigenfuctions of the $\hat{C}\hat{P}$ operator. To show this we first operate on the Kaon states with the $\hat{C}$ operator. We first assume that there is no $\hat{C}\hat{P}$V:

\begin{align*}
\hat{C} \ket{K^{0}(d \bar{s})} = (1)(-1) \ket{\bar{K^{0}}(s \bar{d})} = - \ket{\bar{K^{0}}(s \bar{d})}  \\
\hat{C} \ket{\bar{K^{0}}(s \bar{d})} = (1)(-1) \ket{K^{0}(d \bar{s})} = - \ket{K^{0}(d \bar{s})}  
\end{align*}

\noindent Where we have used the convention that $\hat{C} (q) = 1$ and $\hat{C} (\bar{q}) = -1$. Also, the action of the $\hat{P}$ is given by:
    
\begin{align*}
\hat{P} \ket{K^{0}(d \bar{s})} = \hat{P}(d) \hat{P}(\bar{s})(-1)^{l} \ket{K^{0}(d \bar{s})} = (1)(-1)(-1)^0 \ket{K^{0}(d \bar{s})} = - \ket{K^{0}(d \bar{s})} \\
\hat{P} \ket{\bar{K^{0}}(s \bar{d})} = \hat{P}(s) \hat{P}(\bar{d})(-1)^{l} \ket{\bar{K^{0}}(s \bar{d})} = (1)(-1)(-1)^0 \ket{\bar{K^{0}}(s \bar{d})} = - \ket{\bar{K^{0}}(s \bar{d})} 
\end{align*}

\noindent Where we have used the convention $\hat{P} (fermion) = 1$ and $\hat{P} (anti-fermion) = -1$ as well as $l=0$ because the Kaon is the lowest energy combination of these quarks.

