\section{$\hat{C}\hat{P}$V in Kaon System} 

\subsection{Neutral Kaon Mixing}

As mentioned $\hat{C}\hat{P}$V was first observed in the neutral kaon system. Direct and indirect $\hat{C}\hat{P}$V have been observed but it is found that the process is entirely dominated by the indirect method [Zeng, need better reference]. Essential to these mechanisms is the mixin
g between the neutral Kaon and its anti-particle, corresponding to the states $\ket{K^{0}}$ and $\ket{\bar{K}^{0}}$. These have quark compositions of $d \bar{s}$ and $s \bar{d}$, respectively. 

In interactions involving the strong or EM force, the quantum number strangeness, which tells us the number of strange quarks in a particle, must be conserved. For the weak force it is found that, like parity, this symmetry is not conserved. Due to this many processes forbidden for the strong and EM interactions are allowed through the weak force. This violation is what makes mixing possible. Mixing is the decay of a particle into its anti-particle and can only take place when a particle is its own anti-particle, or if the particles differ by a quantum number which is not conserved by some interaction. This is the case in neutral Kaon mixing, also know as Kaon oscillations. The neutral Kaon and its anti-particle have opposite strangeness but can decay into each other through the strangeness violating weak force. See Fig.(add in feynman digram of M+S pg 289) 

Analogous to the mixing of mass eigenstate quarks to different quark flavours, it is found that the neutral Kaon states do not correspond to eigenfuctions of the $\hat{C}\hat{P}$ operator. To show this we first operate on the Kaon states with the $\hat{C}$ operator. We first assume that there is no $\hat{C}\hat{P}$V:

\begin{align*}
\hat{C} \ket{K^{0}(d \bar{s})} = (1)(-1) \ket{\bar{K}^{0}(s \bar{d})} = - \ket{\bar{K}^{0}(s \bar{d})}  \\
\hat{C} \ket{\bar{K}^{0}(s \bar{d})} = (1)(-1) \ket{K^{0}(d \bar{s})} = - \ket{K^{0}(d \bar{s})}  
\end{align*}

\noindent Where we have used the convention that $\hat{C} (q) = 1$ and $\hat{C} (\bar{q}) = -1$. Also, the action of the $\hat{P}$ is given by:
    
\begin{align*}
\hat{P} \ket{K^{0}(d \bar{s})} = \hat{P}(d) \hat{P}(\bar{s})(-1)^{l} \ket{K^{0}(d \bar{s})} = (1)(-1)(-1)^0 \ket{K^{0}(d \bar{s})} = - \ket{K^{0}(d \bar{s})} \\
\hat{P} \ket{\bar{K}^{0}(s \bar{d})} = \hat{P}(s) \hat{P}(\bar{d})(-1)^{l} \ket{\bar{K}^{0}(s \bar{d})} = (1)(-1)(-1)^0 \ket{\bar{K}^{0}(s \bar{d})} = - \ket{\bar{K}^{0}(s \bar{d})} 
\end{align*}

\smallskip

\noindent Where we have used the convention $\hat{P} (fermion) = 1$ and $\hat{P} (anti-fermion) = -1$ as well as $l=0$ because the Kaon is the lowest energy combination of these quarks. Now we are in a position to determine the eigenstates of $\hat{C}\hat{P}$:

\begin{align*}
\hat{C}\hat{P} \ket{K^{0}} = \ket{\bar{K}^{0}} \\
\hat{C}\hat{P} \ket{\bar{K}^{0}} = \ket{K^{0}} 
\end{align*}

\noindent So we can see that any eigenfunction of the $\hat{C}\hat{P}$ operator will be a linear combination of the two Kaon states:

\begin{align*}
\ket{K^{0}_{1}} = \frac{1}{2} (\ket{K^{0}} + \ket{\bar{K}^{0}}) \\
\ket{K^{0}_{2}} = \frac{1}{2} (\ket{K^{0}} - \ket{\bar{K}^{0}})
\end{align*} 

\noindent Where 1 and 2 are the usual labels given to these states. Now we investigate the action of $\hat{C}\hat{P}$ on these linear combinations

\smallskip

\begin{align*}
\hat{C}\hat{P} \ket{K^{0}_{1}} & = \frac{1}{2} (\hat{C}\hat{P} \ket{K^{0}} + \hat{C}\hat{P} \ket{\bar{K}^{0}}) = \frac{1}{2} (\ket{\bar{K}^{0}} + \ket{K^{0}}) = \ket{K^{0}_{1}} \\
\hat{C}\hat{P} \ket{K^{0}_{2}} & = \frac{1}{2} (\hat{C}\hat{P} \ket{K^{0}} - \hat{C}\hat{P} \ket{\bar{K}^{0}}) =   \frac{1}{2} (\ket{\bar{K}^{0}} - \ket{K^{0}}) = - \ket{K^{0}_{2}} \\
\end{align*} 

In experiment, two Kaon states are observed, a short lived state denoted by $\ket{K^{0}_{S}}$ and a relatively long lived state, $\ket{K^{0}_{L}}$. The lifetimes of these particles are $8.954 \pm 0.004 \times 10^{−11}~$s and $5.116 \pm 0.021 \times 10^{−8}~$s \cite{PDGliveKaonLifetime}. We make the natural assumption that these are the $\hat{C}\hat{P}$ eigenstates just derived. We make the identifications $\ket{K^{0}_{S}} = \ket{K^{0}_{1}}$ and $\ket{K^{0}_{L}} = \ket{K^{0}_{2}}$ and see what this predicts. If $\hat{C}\hat{P}$ is conserved then all the decays of the $\ket{K^{0}_{S}}$ ($CP = 1$) state must be to final products with (CP = 1), and similarly, the decays of $\ket{K^{0}_{L}}$ ($CP = -1$)must be to final products with $CP = -1$. The observed decays for these states are as follows \cite[pg. 292]{Martin+Shaw}:

\begin{eqnarray*}    
K^{0}_S \rightarrow \pi^0 \pi^0 (B = 0.31),  &   &   K^{0}_{S} \rightarrow \pi^{+} \pi^{-} (B = 0.69)\\ [6pt]
K^{0}_L \rightarrow \pi^0 \pi^0 \pi^0 (B = 0.20),   &   &   K^{0}_{L} \rightarrow \pi^{+}  \pi^{-} \pi^0 (B =0.13)  
\end{eqnarray*}    

