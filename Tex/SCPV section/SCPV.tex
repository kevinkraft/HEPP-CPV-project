\documentclass[twocolumn,floatfix,aps,prd,amsmath,amssymb]{revtex4}

\usepackage{epsfig}
\usepackage{color}
\usepackage{graphicx}
\usepackage{float}
\usepackage{listing} %listings
\usepackage{braket}

\begin{document}
% cite:
% SCPV1 ! INTRO (CP3 Seminar)
% SCPV2 SUSY SCPV SO(10)
% SCPV3 Szekeres
% SCPV4 GUT Algebra Baez
% SCPV5 SCPV Group Conditions
% SCPV6 Local Gauge Invariance
\section{Spontaneous Charge Parity Violation} 
A major motivation for creating models with new CPV mechanisms is to explain Baryon asymmetry. This is a huge research area, a simple search on arXiv.org reveals over $500$ papers. The most prevalent theories are based around Super-Symmetry (SUSY) and Spontaneous Charge Parity Violation (SCPV). 

Typically these models are created by looking for gauge symmetries which coincide with Langrangians similar to the Standard Model, but with extra terms that may account for the preference of antimatter to decay into matter.  

SCPV is praised for its ``naturalness" in comparison to regular CPV\cite{SCPV1}. It supposes the possibility to have spontaneous CPV by the vacuum, as opposed to explicit CPV from the CKM matrices. That is, the vacuum is no longer invariant under CP. In a sense this is nice as it may not introduce as many new particles.

Indeed, the Minimal SUSY Standard Model is not compatible with SCPV, and in general it is difficult to incorporate SCPV in SUSY models\cite{SCPV1}, but it has been done. For example in SUSY $\mathrm{SO}(10)$ \cite{SCPV2}.

There are two primary goals to this section. First, to elucidate the use of groups in physics, particularly particle physics. Second, to use this knowledge to understand various SCPV  models beyond the insufficient description given in this introduction.

\subsection{SCPV. Group Theory for Physics}
Some physicists take pride in never having learned group theory and still understanding its applications. This is not an unreasonable point of view, an engineer can launch a rocket without knowing real analysis. 

Undergraduate group theory modules quite commonly focus entirely on groups useful for pure mathematics. This is understandable as it is taught by the math department, but the picture of group theory young physicists may end up with is often quite different to its use in physics.


Part of the reasons groups can be so abstract is that they have very little structure. Even basic physics requires complicated structure. Simply changing coordinates in classical mechanics requires differential geometry, which requires analysis and topology. A \textbf{Group} is a set $G$ with a function $\cdot:G^2\rightarrow G$, having the following properties $\forall g,h,k \in G$
\begin{align*}
\begin{array}{l l l}
(\mathrm{GA}1) & g\cdot h \in G & \text{closure} \\
(\mathrm{GA}2) & \exists e\in G,\phantom{a} e\cdot g = g \cdot e = g \phantom{a}& \text{identity} \\
(\mathrm{GA}3) & \exists g^{-1}\in G, \phantom{a} g^{-1} \cdot g = g \cdot g^{-1} = e & \text{inverse} \\
(\mathrm{GA}4) & (g\cdot h)\cdot k = g\cdot(h \cdot k) & \text{associativity} \\
\end{array}
\end{align*}
Abstract definitions will be avoided in general. To understand this proceed with a useful example. Let $\mathrm{GL}(n,\mathbb{R})$ be the set of all $n\times n$ matrices with real coefficients and non-zero determinant. This forms a group under matrix multiplication.
\begin{itemize}
\item[(GA1)] The product of two $n\times n$ matrices is an $n \times n$ matrix, so it is closed.  
\item[(GA2)] The identity matrix $I$ satisfies $IM=MI=M$.
\item[(GA3)] Non-zero determinant matrices are invertible $M^{-1}M=MM^{-1}=I$.
\item[(GA4)] As real multiplication and addition are associative ($(1+2)+3=1+(2+3)$) Matrix multiplication inherits this property.
\end{itemize}
\textsc{Note:}
\begin{itemize}
\item The set of all real matrices would fail, as zero determinant matrices do not have inverses.
\item Associativity is usually trivial by the definition and not checked.
\end{itemize}
The purpose of this proof was to give some familiar meaning to the abstraction. Detailed proofs shall be avoided in favour of intuition. Note that already  a more powerful structure is present than a group, the field of real numbers, $(\mathbb{R},+,\times)$. This is essentially just two groups glued together, addition and multiplication.

Similarly, every vector space is a group under vector addition. With this it could be said that all of physics uses groups, however a secondary school student does not use differential geometry with Newton's laws. At this stage groups are still useless to the practical physicist.

A \textbf{Transformation Group} is a more useful idea. Let $X$ be a set, $\mathrm{Transf}(X)$ is the set of all one-to-one functions from $X$ to $X$. This is the set of all ways of rearranging $X$. In the context of a finite group $S=\{1,2,...,n\}$, $\mathrm{Transf}(S)$ is just the set of all permutations. 

If $3$-dimensional was modelled space by $\mathbb{R}^3$ the transformation group would not be useful, it would have horribly discontinuous functions that do not relate to intuition about space. What would be useful is:
\begin{center}
The transformations that preserve a property of a space form a group. It is known as the \textbf{invariance group} or the \textbf{symmetry group}.
\end{center}
 This is an extremely general and potent idea. It is also not difficult to prove, so it shall be after the following motivation.
 
Model space again as $\mathbb{R}^3$, transformations should preserve distance. Where Euclidean distance is defined by
\begin{align*}
d(\mathbf{x},\mathbf{y}):= \sqrt{(x_1-y_1)^2+(x_2-y_2)^2+(x_3-y_3)^2}
\end{align*}
 The desired transformation $\mathrm{R}$ has form such that $d(\mathrm{R}\mathbf{x},\mathrm{R}\mathbf{y}) = d(\mathbf{x},\mathbf{y})$. From intuition there are some obvious transformations $\mathrm{R}=I$, the identity matrix, or $\mathrm{R}\mathbf{x} = \mathbf{x}+\mathbf{a}$ a translation in space
\begin{align*}
d(\mathrm{R}\mathbf{x},\mathrm{R}\mathbf{y}) &= d(\mathbf{x}+\mathbf{a},\mathbf{y}+\mathbf{a}) \\
&= \sqrt{\sum_{i=1}^3((x_i+a_i)-(y_i+a_i))^2} \\
&= \sqrt{\sum_{i=1}^3(x_i+a_i-y_i-a_i)^2} \\
&= \sqrt{\sum_{i=1}^3(x_i-y_i)^2} \\
&= d(\mathbf{x},\mathbf{y})
\end{align*}
Now note
\begin{align*}
d(\mathbf{x},\mathbf{y})^2=(\mathbf{x}-\mathbf{y})\cdot(\mathbf{x}-\mathbf{y})
\end{align*}
Where $\cdot$ is the regular dot product. Calling $\Delta\mathbf{x}:=\mathbf{x}-\mathbf{y}$ and using serif font to signify matrix representation
\begin{align*}
\Delta\mathbf{x}\cdot\Delta\mathbf{x} &= \mathsf{\Delta x}^T \mathsf{\Delta x} \\
&= (\mathsf{x}-\mathsf{y})^T(\mathsf{x}-\mathsf{y})
\end{align*}
In other words dot product in an orthonormal basis is a row vector times a column vector. Apply the transformation and assume it preserves distance
\begin{align*}
(\mathrm{R}\mathsf{x}-\mathrm{R}\mathsf{y})^T(\mathrm{R}\mathsf{x}-\mathrm{R}\mathsf{y}) = (\mathsf{x}-\mathsf{y})^T(\mathsf{x}-\mathsf{y})
\end{align*}
The only way this can hold for any $\mathbf{x}$ and $\mathbf{y}$ is if $\mathrm{R}$ is affine, $\mathrm{R}\mathbf{x}=\mathrm{A}\mathbf{x}+\mathbf{a}$ where $\mathrm{A}$ is linear, i.e. a Matrix,
\begin{align*}
(\mathrm{R}\mathsf{x}-\mathrm{R}\mathsf{y})^T(\mathrm{R}\mathsf{x}-\mathrm{R}\mathsf{y}) &= (\mathsf{x}-\mathsf{y})^T(\mathsf{x}-\mathsf{y}) \\
(\mathsf{A}\mathsf{x}+\mathsf{a}-\mathrm{A}\mathsf{y}-\mathsf{a})^T(\mathsf{A}\mathsf{x}+\mathsf{a}-\mathrm{A}\mathsf{y}-\mathsf{a}) &= (\mathsf{x}-\mathsf{y})^T(\mathsf{x}-\mathsf{y})\\
(\mathrm{A}(\mathsf{x}-\mathsf{y}))^T(\mathrm{A}(\mathsf{x}-\mathsf{y})) &= (\mathsf{x}-\mathsf{y})^T(\mathsf{x}-\mathsf{y}) \\
(\mathrm{A}\mathsf{\Delta x})^T(\mathrm{A}\mathsf{\Delta x}) &= \mathsf{\Delta x}^T\mathsf{\Delta x} \\
\mathsf{\Delta x}^T\mathrm{A}^T\mathrm{A}\mathsf{\Delta x} &= \mathsf{\Delta x}^T\mathsf{\Delta x}
\end{align*}
This can only hold if $\mathsf{A}\mathsf{A}^T=I$, so $\mathsf{A}^T = \mathsf{A}^{-1}$. Immediately taking the determinant of both sides gives
\begin{align*}
\det \mathsf{A}\mathsf{A}^T &= \det I \\
\det \mathsf{A} \det \mathsf{A}^T &= 1 \\
\det \mathsf{A} \det \mathsf{A} &= 1 \\
\det \mathsf{A}^2 &= 1 \\
\det \mathsf{A} &\in \{-1,1\}
\end{align*}
This relates to intuition. The determinant of a matrix corresponds to how much it changes a volume. If the determinant of a matrix is $\pm 7$, it will turn a $1$ $m^3$ cube into a $7$ $m^3$ parallelepiped, in order to preserve distance the determinant must have absolute value $1$.

What sort of matrices have determinant $-1$? Consider the example
\begin{align*}
\left( \begin{array}{l l l}
1 &0&0 \\
0&-1&0 \\
0&0&1
\end{array} \right)
\end{align*}
This represents a reflection in through the x-z plane. Likewise for determinant $+1$
\begin{align*}
\left( \begin{array}{l l l}
0&-1&0 \\
1&0&0 \\
0&0&1
\end{array} \right)
\end{align*}
This represents a rotation by $\pi/2$ in the x-y plane. It turns out that all matrices with $\mathsf{A}^T = \mathsf{A}^{-1}$ are either rotations or reflections, together they make the orthogonal group $\mathrm{O}(3)$. A transformation that is an orthogonal matrix plus a translation preserves distance, and the set of all such transformations form a group. Just the usual isometrics of Euclidean space.

Denote the set of rotations, which preserve length and handedness (reflection changes the right hand rule to the left hand rule), as $\mathrm{SO}(3)$, this is also a group.

So far this may seem over simplified but now it is already possible to define groups which preserve much more interesting and relevant properties
\begin{itemize}
\item Galilean Transformations. Preserve Newton's Laws.
\item Unitary Matrices. Preserve complex dot product.
\item Unitary Operators. Preserve normalisation in quantum mechanics. Propagators should have this.
\item Lorrentz Transformations. Preserve the light cone. Equivalently the Minkowski metric.
\end{itemize}
Intuitively, because these preserve something about a space, they form a group. This can greatly simplify some things, they are always invertible, a composition of two transformations still preserves.

 There are many other aspects of group theory that can be useful here. Consider similar matrices, in the context of groups these are called conjugate elements. In the context of rotations they rotate by the same angle (but not the same direction), in the context of Lorrentz Boosts they have the same speed (but not the same direction).
 
 If the reader is interested there is a lot of literature furthering this area, particularly physics-focused is \cite{SCPV3}. In this paper the theory shall not be explained further, aside from the symmetry group theorem proved below. 
\vspace{2mm} \\
\textsc{Theorem:}

{\centering \textit{Consider a set $X$ with a function $f:X\rightarrow Y$, the set of all transformations of $X$ that preserve the function at $x$,\\
$S_f:=\{ T \in \mathrm{Transf}(X) : f(Tx)=f(x), \forall x \in X \}$,\\
form a group called the \textbf{symmetry group} of $f$.}}
\vspace{2mm} \\

For the example of distance in Euclidean space $f(Tx)=f(x)$ corresponded to $d(R\mathbf{x},R\mathbf{y}) = d(\mathbf{x},\mathbf{y})$.
\newpage 

\begin{flushleft}\textsc{Proof:} \end{flushleft}

$(\mathrm{GA}1)$ Let $S,T\in S_f$, then
\begin{align*}
f((ST)x)&=f(S(Tx)) \\
&= f(Sx) \\
&= f(x)
\end{align*}
So $ST$ also preserves $f$, $ST \in S_f$. \\

$(\mathrm{GA}2)$ The identity element of $e$ of $\mathrm{Transf}(X)$ satisfies $f(ex)=f(x)$ by definition of the identity function, so $e \in S_f$.\\

$(\mathrm{GA}3)$  Let $S \in S_f$. Since it is a bijection, an inverse $S^{-1}$ exists. 
\begin{align*}
f(S^{-1}x) &= f(SS^{-1}x) \\
 &= f(ex) \\
  &= f(x)
\end{align*}
$S$ was introduced as it satisfies $f(Sy)=f(y)$ and pick $y=S^{-1}x$. $S^{-1} \in S_f$.\\

$(\mathrm{GA}4)$ is inherited from $\mathrm{Transf}(X)$. \hfill $\square$\\

\textsc{Note}
\begin{itemize}
\item The proof of the Euclidean transformations is incomplete but trivial to extend. Assume $R$ is affine, show it preserves distance.
\item Strictly speaking symmetry groups should be considered group actions. They act on a set $X$, as matrices act on column vectors.
\end{itemize}
\subsection{SCPV. Useful Physical Groups}

In the language of symmetry groups, useful physical transformations including the Gauge-symmetry of the standard model are investigated.

\begin{flushleft}\textit{example 1.0} The Euclidean Isometries. \end{flushleft}

As was shown in the previous section any combination of rotation, reflection and translation preserve Euclidean distance. This can be extended to n-dimensional Euclidean space $\mathbb{E}^n$, here the group of orthogonal transformations and rotations are denoted respectively as $\mathrm{O}(n)$ and $\mathrm{SO}(n)$.

\begin{flushleft}\textit{example 1.1} Galilean Transformations.\end{flushleft}

Solely preserving distance does not say much about physics. Consider Newton's laws in Cartesian coordinates in an inertial frame with column vector representation
\begin{align*}
m \ddot{\mathsf{x}} = \mathsf{F}
\end{align*}
What sort of transformations $\mathsf{Tx}=\mathsf{Rx}+\mathsf{a}$ preserve this equation? Here $R$ is a function of time, for example transforming to a rotating frame
\begin{align*} 
m \frac{d^2}{dt^2}(\mathsf{Rx+a}) &= \mathsf{RF} 
\\
m \frac{d}{dt}(\dot{\mathsf{R}} \mathsf{x} + \mathsf{R} \dot{\mathsf{x}} )+m\ddot{\mathsf{a}} &= \mathsf{RF} \\
m \frac{d}{dt}(\dot{\mathsf{R}} \mathsf{x} + \mathsf{R} \dot{\mathsf{x}} )+m\ddot{\mathsf{a}}&= \mathsf{RF} \\
m (\ddot{\mathsf{R}} \mathsf{x} + 2\dot{\mathsf{R}} \dot{\mathsf{x}}+\mathsf{R} \ddot{\mathsf{x}} )+m\ddot{\mathsf{a}} &= \mathsf{RF} 
\end{align*}
To have the same form this requires, $\dot{R}=0$ and $\ddot{a}=0$. This is equivalent to saying $R$ is a constant change of basis, such as a rotation, and $\mathsf{a} = \mathsf{v}t + \mathsf{b}$, a translation and a constant velocity. In other words
\begin{align*}
\mathsf{Tx} =  \mathsf{Rx} + \mathsf{a}t+\mathsf{b}
\end{align*}
A Galilean transformation is a symmetry group which preserves Newton's laws. It can be readily extended to also preserve distance and time intervals. \\

The goal is to explain the Quantum Field Theory models, Euclidean distance and Newton's Laws are not applicable. They act as an analogy for the Minkowski metric and Quantum Mechanics.

\begin{flushleft}\textit{example 2.0} Lorrentz Transformations. \end{flushleft}
Suppose one wishes to preserve the Minkowski Metric, or equivalently, the light-cone. The transformations that do so are called Lorrentz transformations, essentially defined by
\begin{align*}
\mathsf{L}^T \mathsf{\eta} \mathsf{L} = \mathsf{\eta}
\end{align*}
Extending the transformations to allow for a translation $P^\mu_\nu x^\nu = L^\mu_\nu x^\nu + a^\mu$, these are called Poincaré transformations. The essence of Special Relativity is that the laws of physics are invariant under Poincaré transformation.

\begin{flushleft}\textit{example 2.1} Unitary Transformations. \end{flushleft}
In quantum mechanics the total probability of a time evolving state must always be $1$, otherwise it is simply not a probability. In the context of Hilbert spaces (complex inner product spaces complete under the induced norm), the norm of a state gives its probability, so the norm must be preserved.

Recall in Euclidean space Orthogonal Transformations preserve the norm of a vector, as it preserves the dot product. The complex result is analogous but in the case of operators extends far more, for example
\begin{align*}
\ket{\Psi(t)}=e^{-it\hat{\mathcal{H}} / \hbar}\ket{\Psi(0)}
\end{align*}
The formal solution to the Sch\"odinger equation, the exponential operator on the right is unitary. In general this has quite complicated form. 

If the Hilbert Space is spanned by a finite set $\{\ket{\psi_1},...,\ket{\psi_n} \}$ it is isometrically isomorphic to the inner product space $\mathbb{C}^n$. To preserve the norm here is similar to $\mathbb{R}^n$

\begin{align*}
|U\mathbf{u}|^2 =(\mathsf{Uu})^{*T}\mathsf{Uu} = \mathsf{u}^{*T}\mathsf{U}^{*T}\mathsf{Uu} = \mathsf{u}^{*T}\mathsf{u} = |\mathbf{u}|^2
\end{align*}
This holds $\forall u \in \mathbb{C}^n$ if and only if  $U^{*T}U=I$. This is similar to orthogonal matrices except there is a complex conjugate. The group of all such matrices is denoted $\mathrm{U}(n)$ and those which have positive determinant form a group $\mathrm{SU}(n)$.

\begin{flushleft}\textit{example 3.0} Classical Electrodynamics. \end{flushleft}
\begin{align*}
\nabla \cdot \mathbf{E} &= \frac{\rho}{\varepsilon} \\
\nabla \cdot \mathbf{B} &= 0 \\
\nabla \times \mathbf{E} &= -\frac{\partial \mathbf{B}}{\partial t} \\
\frac{1}{\mu}\nabla \times \mathbf{B} &= \mathbf{j}+\varepsilon\frac{\partial \mathbf{E}}{\partial t}
\end{align*}
As $\nabla \cdot \mathbf{B} = 0$ a theorem shows $\exists \mathbf{A}, \mathbf{B}=\nabla \times \mathbf{A}$ this together with the third equation is
\begin{align*}
\nabla \times \mathbf{E} &= - \partial_t (\nabla \times \mathbf{E}) \\
\nabla \times (\mathbf{E}+\frac{\partial \mathbf{A}}{\partial t}) &=0
\end{align*}
Which in turn gives $\mathbf{E} = -\frac{\partial \mathbf{A}}{\partial t} - \nabla \phi$, where in the time independent case $\phi$ reduced to the electrostatic potential
\begin{align*}
\phi := -\int \mathbf{E} \cdot \mathrm{d}\mathbf{x}
\end{align*}
Consider the following transformation
\begin{align*}
\left( \begin{array}{l}
\mathbf{A} \\ \phi
\end{array} \right) \mapsto
\left( \begin{array}{l}
\mathbf{A}+\nabla \lambda \\
\phi - \partial_t \lambda
\end{array} \right)
\end{align*}
These transformed potentials are also valid
\begin{align*}
\nabla \times (\mathbf{A}+\nabla \lambda) = \nabla\times \mathbf{A} + \nabla \times \nabla \cdot \lambda  = \mathbf{B} + 0 \\
-\partial_t(\mathbf{A}+\nabla \lambda)-\nabla (\phi - \partial_t \lambda) = -\partial_t(\mathbf{A})-\nabla \phi = \mathbf{E}
\end{align*}
As these equations are preserved the gauge transformations form a symmetry group. Just as preserving Newton's laws formed a symmetry group. This is an example of a \textbf{Gauge Transformation}.

More generally a gauge transformation is defined as a \textit{local} transformation that preserves the Lagrangian of a theory. What local means exactly is yet to be defined.

\subsection{SCPV. Gauge Theory}

The reader may be having a hard time seeing how this relates to particle physics. With the benefit of language being developed some light may be shed on what the remainder of this section will contain.

Most of the groups discussed have their place in modern particle physics. Lorrentz invariance and unitarity of symmetry operators obviusly extend from special relativity and quantum mechanics. That is not all, the gauge-symmetry group of the standard model is $\mathrm{SU}(3)\times \mathrm{SU}(2)\times \mathrm{U}(1)$. Many of the popular extensions to the standard Model are also defined by gauge symmetries such as $\mathrm{SU}(5)$, $\mathrm{SO}(10)$ and $\mathrm{O}(16)$ \cite{SCPV4}.

It is possible to state group conditions used for either (a) deducing that a theory can allow SCPV or (b) creating theories which allow for it\cite{SCPVgroups}. Explaining how to use this with the mathematics being developed is the primary goal of the section.


The hierarchy of knowledge required for theoretical physics is unfortunately large. Although the material has attempted to have been presented in a condensed and tangible manner, inevitably some details will be avoided.

In Quantum Mechanics we know overall phase is unimportant, that is $\ket{\psi} \sim \ket{\psi}e^{i\xi}$ where $\xi \in [0,2\pi)$. When making any physical measurement the phase cancels out in the complex inner product. It forms a \textit{global} gauge Symmetry. Phase is still important in Quantum mechanics in the form of phase difference. Consider two energy eigenstates with $a,b\in mathbb{R}$
\begin{align*}
&\ket{\psi} = a\ket{E_1}e^{-iE_1/\hbar} +b \ket{E_2}e^{-iE_2/\hbar} \\
&\ket{\psi}e^{iE_1/\hbar} = a\ket{E_1} +b \ket{E_2}e^{i(E_1-E_2)/\hbar} \\
&\braket{\mathbf{x}} = a^2 \braket{E_1|\mathbf{x}|E_1} +2ab\braket{E_1 |\mathbf{x}|E_2}\cos \theta + b^2 \braket{E_2|\mathbf{x}|E_2}
\end{align*}
So here the phase \textit{difference} $\theta := (E_1-E_2)/\hbar$ appears explicitly in a measurable quantity. Consider $\mathrm{U}(1)$, a "matrix" that acts on one-dimensional complex vectors, i.e., complex numbers. In this case then, both the "matrix" and "vector" can both be represented as complex numbers 
\begin{align*}
|uz|^2=(uz)^* uz = u^*z^* uz = |z^2||u|^2 = |z|^2
\end{align*}
This holds for all $z$ if and only if $u$ has length 1, so $u = e^{i\xi}$ where $\xi \in [0,2\pi)$. So it can be said that the global phase symmetry of Quantum Mechanics can be represented as $\mathrm{U}(1)$. 

Note that $\mathrm{SU}(1)=\mathrm{U}(1)$ in this case, as the norm of a single complex number must be positive. Likewise as $\mathrm{U}(1)$ represents a rotation in the complex plane it must have the same group structure as $\mathrm{SO}(2)$. In mathematical language the groups are isomorphic, written as $\mathrm{U}(1) \cong \mathrm{SO}(2)$.

Thus it is clearer to say that the global phase invariance group of quantum mechanics is isomorphic to $\mathrm{U}(1)$. Frequently it is the case that there is a deeper physical meaning behind symmetries. Merely stating a theory is invariant under $\mathrm{U}(1)$ is not sufficient as one must define how $\mathrm{U}(1)$ acts. For example, it can act globally or locally.
\newpage
Consider Quantum Electrodynamics (QED) the Lagrangian of the theory is invariant under \textbf{local gauge symmetry} of $\mathrm{U}(1)$.
\begin{align*}
\psi(x^1,x^2,x^3,x^4) \mapsto e^{i \xi(x^1,x^2,x^3,x^4)} \psi(x^1,x^2,x^3,x^4)
\end{align*}
Where in cartesian coordinates $(x^1,x^2,x^3,x^4)=(x,y,z,ct)$. In shorthand
\begin{align*}
\psi(x) \mapsto  \tilde{\psi}(x)=e^{i \xi(x)} \psi(x)
\end{align*}
Where x:=(x,y,z,ct) This inherently insists that phase is irrelevant at any point. 


It was noted that more structure than merely a group is being used  concerning physical groups. Fields, vector spaces, and vector calculus to name a few. Primarily calculus, an inherently geometric structure, appears everywhere in physics. The real scalar function $\xi(x)$ must be differentiable.

This differentiability calls for the structure of a \textbf{Lie Group} (acting on a set). A Lie Group is simply a group where the group operation and taking inverses is differentiable. For a simple example consider the group of addition on $\mathbb{R}^n$.
\begin{align*}
&\cdot(x,y) = x+y &\mathrm{inv}(x) = -x
\end{align*}
Both of these functions are differentiable in their arguments so $(\mathbb{R},+)$ forms a Lie Group. Likewise other familiar groups $\mathrm{SU}(n)$,$\mathrm{O}(n)$ and $\mathrm{GL}(n,\mathrm)$ all form Lie groups.

To define calculus on non-Euclidean spaces one needs the structure of differential manifolds. The details of which will have to be omitted but some intuitive and useful properties are worth discussing.

The dimension of a differential manifold is related to our understanding of dimension of space. The surface of a sphere is a two-dimensional space and as expected forms a two-dimensional manifold. This means, for the most part, the sphere can be parametrised by two numbers $(\theta,\phi)$.

$\mathrm{O}(n)$, $\mathrm{U}(n)$ and $\mathrm{SU}(n)$ have manifold dimension $\frac{1}{2}n(n-1)$, $n^2$ and $n^2-1$ respectively. So these objects are no longer purely to be considered as groups but also as abstract smooth geometric spaces.

Returning the example of QED's local gauge symmetry and comparing with the global gauge symmetry of QM
\begin{align*}
\ket{\psi} &\mapsto  \ket{\tilde{\phi}}=e^{i \xi} \ket{\psi(x)} \\
\psi(x) &\mapsto  \tilde{\psi}(x)=e^{i \xi(x)} \psi(x)
\end{align*}
Both symmetries are isomorphic to $\mathrm{U}(1)$ but in very different contexts. In QM it may be regarded simply as a group, but in QED it effects the lagrangian much differently.

The local gauge symmetry of the strong force is $\mathrm{SU}(3)$ and for the weak force it is $\mathrm{SU}(2)$. Like $U(1)$ was represented by one parameter which was a function of space-time position $\xi(x)$, $\mathrm{SU}(2)$ and $\mathrm{SU}(3)$ have $2^2-1=3$ and $3^3-1=8$ respectively.

Notice peculiarly these these are the amount of force carriers in each theory.

What do these symmetries mean physically? According to \cite{SCPV6}, they are like the the potentials $\mathbf{A}, \phi$ for classical electromagnetism; merely a tool to make the mathematics easier. In the same way as one could use $\mathbf{E}$ and $\mathbf{B}$ without the degeneracy of $\mathbf{A}$ and $\phi$, in principle one could formulate QED completely independent of phase, even without complex numbers.
However the difference between usability would be extreme.


A full treatment of local gauge theory for physics can be found here \cite{SCPV6} and more on Lie-algebras and differential manifolds related to physics can be found at \cite{SCPV3}. 

The gauge symmetries of the standard model are written as the product group $\mathrm{SU}(3)\times \mathrm{SU}(2)\times \mathrm{U}(1)$. A product group is formed in the same way $\mathbb{R}^2$ with vector addition is a product group of $\mathbb{R}\times\mathbb{R}$. This can also be considered a project manifold and indeed it tells us the number of force carriers there are in each theory. Knowing the dimension of the SUSY theory $\mathrm{SU}(5)$ as $5^2-1 = 24$ then tells us the number of force carriers in this theory.

One must call it at a day at some point, actually dealing with Lagrangians is a much more time consuming and worth-while task, but the language developed shall suffice to investigate SCPV.

\subsection{SCPV. Group Theoretic Conditions.}

\newpage
pig

\end{document}
