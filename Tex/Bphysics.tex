%\RequirePackage{fixltx2e}
%\documentclass[floatfix,aps,prd,amsmath,amssymb]{revtex4}
%\usepackage{graphicx}
%\usepackage{caption}
%\usepackage{subcaption}
%\captionsetup{compatibility=false}
%\usepackage[noabbrev,capitalise]{cleveref}
%\usepackage{braket}
%\begin{document}

\section{B mesons and CPV}
\vspace{-1.0em}
\begin{center}
\tiny{\textit{John Ronayne}}
\end{center}

The neutral B meson is composed of the 1st generation down quark and a 3rd generation bottom quark. While this doesn't contrast the neutral Kaon oscillations to major extent when covering the preliminaries with the benefit of the CKM theory we see that the B mesons produce a much more dramatic effect.Jumping ahead we define the linear combinations of the B meson system with egenfunction of $\hat{C}\hat{P}$,
\[\ket{B^{0}_{1}}=\frac{1}{\sqrt{2}}(\ket{B^{0}(d\bar{b})}+\ket{\bar{B}^{0}(\bar{d}b)},\]
\[\ket{B^{0}_{2}}=\frac{1}{\sqrt{2}}(\ket{B^{0}(d\bar{b})}-\ket{\bar{B}^{0}(\bar{d}b)},\]
So including mixing we have the two mass eigenstates,
\[\ket{B_{L}}=p\ket{B^{0}_{1}}+q\ket{{B}^{0}_{2}},\]
\[\ket{B_{H}}=p\ket{B^{0}_{1}}-q\ket{{B}^{0}_{2}},\]
We have adopted the terminology of L for light and H for heavy in relation to the relative masses. We can define mass eigenvalues using the CKM matrix,
\[\frac{q}{p}=\frac{V^{*}_{td}V_{td}}{V_{td}V^{*}_{td}}=e^{-i2\beta}\]
From here we can develop on a few aspects of the B meson system. To study some direct  \textit{CP} violations we can look at the magnitude  $\left| \frac{q}{p} \right|$. The $\beta$ phase is associated to the mixing and interference and is of interest in experimental measured Indirect  \textit{CP} violation. First the time dependent state must be considered to provide a basis on which the coherent states may be experimentally measured. 


When experimental data is analyzed one B meson is reconstructed fully (this includes the flavor, B or $\bar{B}$) we call this one $B_{rec}$ with decay time $t_{rec}$ while its sibling is inferred from the decay of the $B_{rec}$ and is called $B_{tag}$ with a $t_{tag}$ decay. The time-depended Asymmetry due to \textit{CPV} in terms of $\Delta{t} = t_{rec}-t_{tag}$ and the number of events N is,
\[\it{A}_{CP}(\Delta t)= \frac{N(B^{0}_{tag},\Delta t)-N(\bar{B}^{0}_{tag},\Delta t)}{N(B^{0}_{tag},\Delta t)+N(\bar{B}^{0}_{tag},\Delta t)},\]



The states $\ket{B^{0}}$ or $\ket{\bar{B}^{0}}$ evolve from t=0 to a pure state of $\ket{B_{phys}^{0}}$ or $\ket{\bar{B}^{0}_{phys}}$ at great t values. 
These states have Decay rate eigenstates as follows,
\[\ket{B(t)}= g_+(t)\ket{B^0}+\left(\frac{q}{p}\right)g_{-}(t)\ket{\bar{B}^0}\]
\[\ket{\bar{B}(t)}= \left(\frac{p}{q}\right)g_-(t)\ket{B^0}+g_{+}(t)\ket{\bar{B}^0}\]
with,
The complete B meson Amplitude for the decay from a $\Upsilon(4s)$ to the final product of $f_{tag}$ or $f_{rec}$ is,
 \[g_{\pm}(t)=\frac{1}{2}(e^{-i\frac{\Delta m_d \Delta t}{2}}e^{-i\frac{\Delta\Gamma \Delta t}{4}}\pm e^{+i\frac{\Delta m_d \Delta t}{2}}e^{+i\frac{\Delta\Gamma \Delta t}{4}})\]
Which can be shown to equal [ref],
\[g_{+}(t)=e^{-iMt}e^{-\Gamma\frac{t}{2}}\cos(\Delta m_d t/2) \mbox{ and }g_{-}(t)=e^{-iMt}e^{-\Gamma\frac{t}{2}}i\sin(\Delta m_d t/2),\]
when $\Delta\Gamma \approx 0 \mbox{ , }\Delta m_d = 0.502 \pm 0.007 ps^{-1}$, $\Gamma = \frac{1}{\tau_{B^0}}$ and $M=\frac{1}{2}(M_H+M_L)$.

The account for a minus due to the antisymmetric properties of the B mesons which are produces in a P wave state. To fully expand on this we would require the individual amplitudes for our initialized mesons to decay to either final state. So one meson has the probability to decay to $f_1$ at $t_1$ and the other at $t_2$ to decay to $f_2$. We mentioned $\Delta t$ before but it is of course just $\Delta t = t_1 -t_2$ and $T= t_1 +t_2$.

\[A_{1,2}=\bra{f_{1,2}}\it{H_{Weak}}\ket{B^{0}} \mbox{ and } \bar{A}_{1,2}=\bra{f_{1,2}}\it{H_{Weak}}\ket{\bar{B}^{0}}\]
%\[\it{A} = \braket{f_{tag}|B^{0}_{phys}(t_{tag})} \braket{f_{rec}|\bar{B}^{0}_{phys}(t_{rec})} -\braket{f_{tag}|\bar{B}^{0}_{phys}(t_{tag})} \braket{f_{rec}| B^{0}_{phys}(t_{rec})} \]

A result we can test is ideal. What we can test then is the time-dependent rate of producing a particular decay by the term $\it{F}=\frac{\partial\Gamma}{\partial t}$. 
\[\it{F}(T,\Delta t)=e^{-\Gamma\left|\Delta t\right|}\left| a_{+}g_{+}(\Delta t)+ a_{-}g_{-}(\Delta (t) \right|^{2})\]

the values of $a_+$ and $a_-$ correspond to,

\[ a_{+}=\bar{A}_{tag}A_{rec}-A_{tag}\bar{A}_{rec} \mbox{ and }a_{-}=\left(-\frac{q}{p}\bar{A}_{tag}\bar{A}_{rec}+\frac{p}{q}A_{tag}A_{rec}\right)\]

The use of this shall now be seen but the importance to the understanding of $\it{CP}$ violation is immense. T may be replaced by $\Delta t$ which is a fundamental design capability of detector to measure, this means the physical distance of vertexes of each mason are found. As well as this the coherent states produced mean that the flavor of the particle we detect is fundamentally linked to the paired partner, its tag.

In a time dependent system of coherent states we often define physical properties such a spin or polarization when dealing with light of electrons. Yet the properties ,albeit different follow the same rules. For a system of B mesons the observables we which to examine and the apparent natural asymmetries are found in the eigenstates of either flavor or $\it{CP}$. 

In the case of flavor eigenstates we base our principle of mixing of coherent states. When a B meson is detected there is a quantum game of chance at play. The $B_{rec}$ is what is seen and we will find it ,through the decay product, to be either a $B^0$ or $\bar{B}^0$ and it partner $B_{tag}$ is found to be the opposite flavor, so  chronologically $\bar{B}^0$ or $B^0$. This is what is know as an unmixed event. The time-dependent rate of decay is thus,
\[F_{unmix}(\Delta t) \propto e^{-\Gamma\left|\Delta t\right|}(1 + cos(\Delta m_d \Delta_t),\]
However there is the other probability. One which we have show to occur is the mixing of of the two B's. Similar to spin flipping in a beam of neutrons or protons we have the flavor being flipped. The $B_{rec}$ having been detected in a $B^0$ or $\bar{B}^0$ state means that the $B_{tag}$ is in the exact same $B^0$ or $\bar{B}^0$ state. Giving a time-dependent rate of decay of,
\[F_{mix}(\Delta t) \propto e^{-\Gamma\left|\Delta t\right|}(1 - cos(\Delta m_d \Delta_t),\]
The oscillations of the B meson system is directly related to the Mixing asymmetry.
\[A_{mix}(\Delta t) = \frac{F_{unmix}(\Delta t)-F_{mix}(\Delta t)}{F_{unmix}(\Delta t)+F_{mix}(\Delta t)} = cos(\Delta m_d\Delta t) \]

Now we study the eigenstate of $\it{CP}$. Unlike before when the concetration was in the relationship between the two decay products of $B_{rec}$ and $B_{tag}$, we focus purely on our $B_{tag}$ as the CP eigenstate of $B_{rec}$ is assumed. Thus the amplitudes are found from the probable modes of decay. New decay amplitudes are defined on the basis of the final state is accessible due to a $\it{CP}$ operation.
\[A_{f_{CP}}==\bra{f_{CP}}\it{H_{Weak}}\ket{B^{0}} \mbox{ and } \bar{A}_{f_{CP}}=\bra{f_{CP}}\it{H_{Weak}}\ket{\bar{B}^{0}}\]
In scenario (a) the $B_{tag}$ is decaying via a $A_{f_{CP}}$ or $\bar{A}_{f_{CP}}$ while $B_{rec}$ decays as $A_{f_{CP}}$ and in scenario (b) $B_{tag}$ is decaying via a $A_{f_{CP}}$ or $\bar{A}_{f_{CP}}$ while $B_{rec}$ decays as $\bar{A}_{f_{CP}}$. 

Two rates of decay arise on the pretense of the original $B_{tag}$ flavor, these are

\[F(B_{tag}=B^{0},\Delta t) \propto
 e^{-\Gamma |\Delta t|}
\left[1+\frac{1-\left|\frac{q}{p}\frac{\bar{A}_{f_{CP}}}{A_{f_{CP}}}
\right|^{2}}{1+\left|\frac{q}{p}\frac{\bar{A}_{f_{CP}}}{A_{f_{CP}}}\right|^{2}}\cos(\Delta m_{d}\Delta t) - 
\frac{2\it{I}m\frac{q}{p}\frac{\bar{A}_{f_{CP}}}{A_{f_{CP}}}}
{1+\left|\frac{q}{p}\frac{\bar{A}_{f_{CP}}}{A_{f_{CP}}}
\right|^{2}}
\sin(\Delta m_{d} \Delta t)\right]\]
and,
\[F(B_{tag}=B^{0},\Delta t) \propto
 e^{-\Gamma |\Delta t|}
\left[1+\frac{1-\left|\frac{q}{p}\frac{\bar{A}_{f_{CP}}}{A_{f_{CP}}}
\right|^{2}}{1+\left|\frac{q}{p}\frac{\bar{A}_{f_{CP}}}{A_{f_{CP}}}\right|^{2}}\cos(\Delta m_{d}\Delta t) + 
\frac{2\it{I}m\frac{q}{p}\frac{\bar{A}_{f_{CP}}}{A_{f_{CP}}}}
{1+\left|\frac{q}{p}\frac{\bar{A}_{f_{CP}}}{A_{f_{CP}}}
\right|^{2}}
\sin(\Delta m_{d} \Delta t)\right]\]
Again we find the time-dependent asymmetry to be,
\[A_{mix}(\Delta t) = \frac{F_{B_{tag}=B^0}-F_{B_{tag}=\bar{B}^0}}{F_{B_{tag}=B^0}+F_{B_{tag}=\bar{B}^0}} = \frac{1-\left|\frac{q}{p}\frac{\bar{A}_{f_{CP}}}{A_{f_{CP}}}
\right|^{2}}{1+\left|\frac{q}{p}\frac{\bar{A}_{f_{CP}}}{A_{f_{CP}}}\right|^{2}}\cos(\Delta m_{d}\Delta t) - 
\frac{2\it{I}m\frac{q}{p}\frac{\bar{A}_{f_{CP}}}{A_{f_{CP}}}}
{1+\left|\frac{q}{p}\frac{\bar{A}_{f_{CP}}}{A_{f_{CP}}}
\right|^{2}}
\sin(\Delta m_{d} \Delta t)\]

While the shape of the decay rates are in someway unique and a definite property of the asymmetry we wish to observe, the real determination we seek is from the influence of $\Delta t$. This contributes to the amplitude, and the predicted rate which can be observed produced and detected in particle accelerators. As the measurement of such can then be compared back to eqn.[?] to find the $\it{CP}$ angle $\beta$.

%@article{PhysRevD.68.034010,
 % title = {Impact of tag-side interference on time-dependent         \textit{CP}       asymmetry measurements using coherent ${B}^{0}{B}^{0}$ pairs},
%  author = {Long, Owen and Baak, Max and Cahn, Robert N. and Kirkby, David},
%  journal = {Phys. Rev. D},
%  volume = {68},
 % issue = {3},
 % pages = {034010},
 % numpages = {11},
 % year = {2003},
 % month = {Aug},
 % doi = {10.1103/PhysRevD.68.034010},
 % url = {http://link.aps.org/doi/10.1103/PhysRevD.68.034010},
 % publisher = {American Physical Society}
%}

%http://arxiv.org/pdf/hep-ph/9806471v1.pdf
\subsection{BaBar}
While the decay of the Z boson at LEP was initially to study the daughter B meson particles and their asymmetry a detailed study required a more dedicated experiment and one which B meson were the sole product. What are called 'B-factories' were designed. BaBar (or $B\overline{B}$ in Stanford and Belle in Japan are experiments which create these B meson in large quantities (by large we mean 10 per second). We  focus our attention on the BaBar experiment and how it produces and detects the B mesons. The linear accelerator injects two high energy beams (electrons and positrons) into a circular collider PEP-II. Unlike most colliders the two beams are accelerated to different energies. In particular the electron beam has an energy of 9GeV and the positron beam has an energy of $3.1GeV$. The energy at the center of mass is correspondingly 

\begin{figure}[h]
\centering
\includegraphics[width=0.75\textwidth]{figs/BBD.jpg}
\caption{The BaBar Detector Schematics}
\label{BBD}
\end{figure}



[working of C.o.M energy ~10.58GeV]

At 10.58GeV it is capable of producing an $\Upsilon(4s)$ which quickly decays to either a $B^+B^-$ or $B^0\overline{B}^0$ pair. Since the Upsilion rest mass is only just enough to create the pair the asymmetric energy beam provides the momentum to separate the pair.
[Fiddlieman Diagram of decay and production]
The uniqueness of this imbalance in momentum gives B mesons addition momentum upon creation relative to the laboratory centre of mass frame. The benefit is that the distances the B mesons travel are the measureable and due to time dilation induced by their high velocity the lifetime of the B and $\overline{B}$ mesons can be determined and compared to considerable accuracy. [1]. The calculauble separation that the B’s travel are on average 260um [2]. The Upsilons decays to the $B^0 \overline{B}^0$ pair a quarter of  the total hadronic cross section [citation+decay table]. Thus the luminosity of the beam need to be significantly high to have a reliable amount of data.
The detector itself is the heart of the experiment and where some marvelous engineering and physics has been implemented to preform a detailed study of very specific decays with incredible high fidelity. The $\Upsilon(4s)$ is produced at the interaction point as shown in figure[??] and while the decay products are sweep through the detector via their own momentum or through the 1.5T magnetic field produced by the superconducting coils. 

\begin{figure}[h]
\centering
\includegraphics[width=0.75\textwidth]{figs/dt.JPG}
\caption{$\Upsilon (4s)$ decay illustrating the crucial experimental parameter $\Delta t$}
\label{BBD}
\end{figure}


\begin{itemize}
\item The Silicon Vertex Tracker(SVT) consists of five-layers of double sided silicon orientated as interlacing wafers along the inner most part of the detector. Residing only 3.2cm from the center of the 2.7cm radius beam pipe it receives  high fidelity detention of high energy charged particles and low energy e+ e- pairs to an resolution of 10 microns.[a1] The primarily purpose is to contribute to the measurement of the angular and vertex information (the z  and r - $\phi$)of each track which corresponds to the $\Delta z$ resolution where the z-axis is the parallel to the beam. The design had taken into consideration the asymmetric beam energy with the length of each layer increasing relative to the $x-y$ plane and shifted further down the z axis (along the higher energy beam path). [diagram2]. Another design consideration to put in place was the radiation protection. Given the high luminosity of the beam and higher probability of Coulomb scattering inside the pipe the electronic were made to withstand the near $250kRad$ per year [a.1].


%[a1]http://hep.ucsb.edu/people/claudio/Vancouver.pdf
%[a2]http://www.slac.stanford.edu/cgi-wrap/getdoc/slac-pub-11695.pdfm


\item The Drift Chamber(DC) is responsible for measurements of the moment of produced particles as well as information on identifying particles that lose energy inside (the value of $dE/dt$). It extends from the edge of the SVT (22cm) to a radial distance of 80cm . Inside are 7104 small cells connected to tungsten-rhenium sense wires. These wires have a 2kV potential across them and are thick enough to detect the ionized particles while keeping scattering minimal . The chamber contains a gas mixture of 20$\%$ Isobutane and 80$\%$ Helium allowing the atoms to be ionized and detected by the sense wires. [a4]


%http://www.phys.hawaii.edu/superb04/talks/Kelsey.pdf


\item The Detector of Internally Reflected Cerenkov light (DIRC) takes advantage of the Cerenkov radiation. This is a process whereby charged particle traveling at speeds greater than the local speed of light in a medium. Silica rods are placed parallel to the beam pipe surround the outer wall of the Drift chamber. The large index of refraction ($n= 1.474$) means that the speed of light in that medium is only $0.68c$ and a critical angle of about $43^{\circ}$. As charged particle enter the the silica rod and are of sufficient velocity they produce Cerenkov radiation and if it happens to be inside the critical angle it will be internally reflected along the rod to the back of the detector where they enter the "standoff box" containing water and then detected by a ring (correctly speaking a toroidal) of Photomultiplier. Reflecting "light catching cones" capture the light that might otherwise miss the PMT's active surface.  The energy and angle incident angle of each photon is reconstructed during analysis. The primary reason to have a system as such installed is to determine between Kaons and Pions between 0.5 and 4.5$GeV$, this is part of the particle Identification (PID) system. 


%[a3]http://www.slac.stanford.edu/cgi-wrap/getdoc/slac-pub-8080.pdf
\item The EM Calorimeter's (EMC) purpose is to measure the energy and angular resolution between $20MeV$ and $9GeV$. The high energy bar provides detection of the more energetic electrons, muons and photon. Slow moving neutral particles such as the $\pi^0 \mbox{ and the } \eta^0$ will also be detected here.The particles rest mass energy is completely absorbed here due to interactions with the dense material. 6580 Csl(TI) (crystals grown from Csl and doped with 0.1$\%$ Thallium ) trapezoidal crystals are carbon fiber modules .Each Crystal is roughly the dimensions of a standard Rubix cube. Physically it extends a 0.92m to 1.27m and parallel to the pipe 2.3m downstream and 1.5m upstream from the 9GeV beam. At the longer arm length there is an endcap of crystals to facilate the off vertical production of end products, moreover each module is angle towards the central interaction point.

%http://iopscience.iop.org/1742-6596/160/1/012004/pdf/1742-6596_160_1_012004.pdf
\item The Instrument Flux Return (IFR) is separated from the EM Calorimeter by the superconducting coil and the farthest detection instrument from the Interaction point. Here Muons and neutral hadrons (from the light $\pi^0$ to the heavy $K^{0}_{L}$ are detected with the use of the large iron structure needed as the magnetic return yoke. It is segmented into 19 hexagonal layers from 1.8m to 3m from the beam pipe. Between each layer is a single gap resistive plate chamber (RPC) which serve the purpose of detecting ionizing particles such as muon. Muons themselves are identified on the criterion of penetrating ever layer of iron, some slower Muons are identified in the RPCs. 
\end{itemize}
%http://www.slac.stanford.edu/cgi-wrap/getdoc/slac-r-457.pdf
A process of particle Reconstruction and recognition from electronic signatures produced in the detector to the vast system of filtering and discriminating between background process using algorithms and triggering [?] produce reveals the results to compare to our theory.[expand on this]>> The time difference $\Delta t = \frac{\Delta z}{\left< \beta\gamma\right>} c$ is obtained from the measured $\Delta z=z_1 - z_2$ and average boost and average boost $\left<\beta\gamma\right>$. Since the boost is known to good precision, the Dz measurement dominates the $\Delta t$ resolution.
\subsection{Experimental Evidence of $\it{CPV}$ in B mesons}
Testing B meson oscillations in the BaBar detector has been accomplished [??] through the di-lepton and semi-leptonic events. Examples of such a decay are 
$B^{0}\rightarrow D^{*-}l^{+}\nu$. By registering charged leptons the flavour of the decayed B meson can be calculated. Data taken at BaBar are ploted to a likelihood fit. The likelihood fit takes into account the probability of mismatched tags, the resolution of $\Delta t$ and background. In Fig[.a] the unmixed decays are shown where the B meson pair are of opposite flavor while B meson pairs that have identical flavors in the mixed state are found in Fig[.b]. Finally the  time dependent asymmetry of the two is found in Fig[.c] clearly exhibiting a cosine wave form as predicted, almost akin to the an inference pattern an undergraduate might observe in laboratories. The frequency of the B meson oscillations is 80 GHz. Noticeably the amplitude of the Mixed state is significantly lower than unmixed, which is to be expected[detail]. The neutral mixing frequency has been measure to high accuracy at BaBar, the inclusive dilepton sample to be $\Delta m_d = 0.493 \pm 0.012(stat) \pm 0.009(syst) ps^{-1}$.

%@article{PhysRevLett.88.221803,
  %title = {Measurement of the ${\mathit{B}}^{0}-{\overline{\mathit{B}}}^{0}$ Oscillation Frequency with Inclusive Dilepton Events},
 % author = {Aubert, B. et. al},
 %journal = {Phys. Rev. Lett.},
  %volume = {88},
  %issue = {22},
  %pages = {221803},
  %numpages = {7},
  %year = {2002},
  %month = {May},
  %doi = {10.1103/PhysRevLett.88.221803},
  %url = {http://link.aps.org/doi/10.1103/PhysRevLett.88.221803},
  %publisher = {American Physical Society}
%}
 \begin{figure}[h]
\centering
\includegraphics[width=0.5\textwidth]{figs/Flavourosscilatons.JPG}
\caption{(a) Unmixed flavored time dependent decay rate (b) Mixed flavored time dependent decay rate (c) Asymmetry of mixing in flavor eigenstates}
\label{BBD}
\end{figure}

The CP eigenstate decay is on of the most important measurements studied. Much of the theory on CKM matrixes are easily verified by these means. We can approximate our amplitude in eqn[] to something like,
\[\frac{q}{p}\frac{\bar{A}_{f_{CP}}}{A_{f_{CP}}}=\eta_{f_{CP}}e^{2i\beta}, \left|\frac{q}{p}\frac{\bar{A}_{f_{CP}}}{A_{f_{CP}}}\right|=1, \it{I}m\frac{q}{p}\frac{\bar{A}_{f_{CP}}}{A_{f_{CP}}}=-\eta_{f_{CP}}\sin2\beta\]
\[A_{CP}=-\eta_{f_{CP}}\sin2\beta\sin(\Delta m_d \Delta t)\]
To obtain a value for the $\beta $ term we focus on two elements. The $\Delta t$ becomes important and precision measurements of the asymmetry. The Decay of interest here is $B\rightarrow J/\Upsilon K$ [fiddleman diagram here], this is a favorable decay due to the large branching fractions $(\sim 10^{-4})$ and narrow resonance allowing a clean signal above background[ref:recentmes]. In figure[lableit] the asymmetry manifesting due to CP violation is clear visible with the data and likelihood fit corresponding predicted characteristics. The result for $\sin (2\beta) =0.722\pm 0.040_{stat} \pm0.023_{syst}$. Moreover due to ambiguities in the angle $\beta$  and further time depend analysis on the angular decay, $\cos(2\beta)$ is determined an in agreement with the standard model.[ererfe]

 \begin{figure}[h]
\centering
\includegraphics[width=0.36\textwidth]{figs/cpf.JPG}
\caption{(a) Time Distribution in CP odd, $K_{S}$ (b) Raw Asymmetry with likelihood plot (c) Time Distribution in CP even, $K_{L}$  (d)}
\label{BBD}
\end{figure}
%[ererfe]
%@article{PhysRevD.71.032005,
 % title = {Time-integrated and time-dependent angular analyses of $B$\rightarrow${}J/$\psi${}K$\pi${}$: A measurement of $\mathrm{cos}{}2$\beta${}$ with no sign ambiguity from strong phases},
%  author = {Aubert, B.},
%collaboration = {The BABAR Collaboration},
%  journal = {Phys. Rev. D},
%  volume = {71},
%  issue = {3},
%  pages = {032005},
%  numpages = {30},
%  year = {2005},
%  month = {Feb},
%  doi = {10.1103/PhysRevD.71.032005},
%  url = {http://link.aps.org/doi/10.1103/PhysRevD.71.032005},
%  publisher = {American Physical Society}
%}


%@article{Sciolla:2005kz,
%      author         = "Sciolla, Gabriella",
%      title          = "{Recent measurements of sin2b at BABAR}",
%      collaboration  = "BaBar Collaboration",
%      journal        = "Nucl.Phys.Proc.Suppl.",
%      volume         = "156",
%      pages          = "16-20",
%      doi            = "10.1016/j.nuclphysbps.2006.03.054",
%      year           = "2006",
%      eprint         = "hep-ex/0509022",
%      archivePrefix  = "arXiv",
%      primaryClass   = "hep-ex",
%      reportNumber   = "BABAR-PROC-05-030, SLAC-PUB-11485",
%      SLACcitation   = "%%CITATION = HEP-EX/0509022;%%",
%}
%@article{Boos:2004xp,
  %    author         = "Boos, Heike and Mannel, Thomas and Reuter, Jurgen",
     % title          = "{The Gold plated mode revisited: Sin(2 beta) and B0
     %                   ---&gt; J / Psi K(S) in the standard model}",
     % journal        = "Phys.Rev.",
    %  volume         = "D70",
    %  pages          = "036006",
    %  doi            = "10.1103/PhysRevD.70.036006",
     % year           = "2004",
     % eprint         = "hep-ph/0403085",
    %  archivePrefix  = "arXiv",
     % primaryClass   = "hep-ph",
      %reportNumber   = "SI-HEP-2004-04, TTP03-23",
    %  SLACcitation   = "%%CITATION = HEP-PH/0403085;%%",
%}


%\[\left| \frac{\bar{A}_{\bar{f}}} {A_{f}} \right|\neq 1 \]
%\[A_{CP}=\frac{1-\left|\frac{\bar{A}_{\bar{f}}}{A_f}\right|^2}{1+\left|\frac{\bar{A}_{\bar{f}}}{A_f}\right|^2}\neq 2\]
Since we have obtained a $\beta$ term we may now look towards finding a relevant $\gamma$ term to have out unitary triangle near complete. The CKM phase arises from the $V_{ub}$ term in the interferences of $b\rightarrow c \mbox{ and } b \rightarrow u$ transitions, note that the $V_{cb}$ term is phase-less. We construct six possible Feynman diagrams for this process,
\[B^0 \rightarrow D^{*\mp}\pi^{\pm} or B^0 \rightarrow D^{*\mp}\rho^{\pm} \]
The first two are seen in \label{pBGD}. The charged products are clean pointers to a flavor of our $B_{rec}$ meson and we can, as per usual, infer the $B_{tag}$. As per the methods we have used in deriving eqn[??] we follow [ref] to construct the $\it{CP}$ asymmetry term, in terms of j, where j is only a permutation between the various end products of decay.
\[\it{A}^{j}_{\it{CP}}(\Delta t) = \frac{2r^j}{1+\left[r^j\right]^2}\sin(2 \beta +\gamma)cos(\delta^j)\sin(\Delta m_d \Delta t)\]
Since the decay from $\bar{B}^0 \rightarrow D^{*+}\pi^{-}$ is favored via CKM mixing amplitudes over what is called the double-CKM-suppressed decay $B^0 \rightarrow D^{*+}\pi^{-}$ due to two low amplitude mixing terms we see a high sensitivity to the CP angle $\gamma$  \label{kino1} is plot and likelihood fit of the decay distribution. Since measurement of $\gamma$ are still in their infancy the present value is $\gamma = 78^{\circ}\pm12^{\circ}$. We will ignore a discussion on obtaining a value for $\alpha$ since no diirect measurement have been successful made no b transition will produce such a phase. But we can find a value intrinsically by the relations $\sin2\alpha = - \sin(2\beta +2\gamma)$.

%http://cds.cern.ch/record/1106345/files/CERN-THESIS-2008-044.pdf
%MA BAAK thesis/paper
 \begin{figure}[h]
\centering
\includegraphics[width=0.6\textwidth]{figs/kino.JPG}
\caption{(a) Time dependent decay for $\bar{B}^0 \rightarrow D^{*-}\pi^{+}$ (b) Time dependent decay for $B^0 \rightarrow D^{*+}\pi^{-}$. If no double-CKM-Suppression were evident the dashed line would be the amplitude.}
\label{kino1}
\end{figure}

 \begin{figure}[h]
\centering
\includegraphics[width=0.8\textwidth]{figs/gam.JPG}
\caption{Convert these puppies to le fiddlemans}
\label{pBGD}
\end{figure}


We have assumed with further explanation the three types of CP violation in this section. Direct being the spontaneous decay via a $\it{CP}$ operation. Indirect being a pure asymmetry in the mixing of B-states and the the combination of the two. Of these there purely indirect has never been observed as the affects are minuscule and dominated in region of high background.


The full range of angle measurement obtainable by B meson decay are shown in figure [??] .Remarkable the validation of this particular Unitary triangle has stood up to the rigirous measurements made at BaBar and pave a path for the further develepodment of new physics to provide the concrete understanding that lie at the fundamental mechanism of this asymmetry.
 \begin{figure}[h]
\centering
\includegraphics[width=0.8\textwidth]{figs/trig.JPG}
\caption{The B meson Unitary triangle}
\label{BBD}
\end{figure}

%http://pprc.qmul.ac.uk/~bona/ulpg/cpv/lecture3.pdf


%\end{document}
