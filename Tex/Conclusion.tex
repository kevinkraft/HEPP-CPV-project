\section{Conclusion}

This review has presented the current state of CPV through the CKM theory, experimental evidence, group formalism and cosmological models. 
The CKM mechanism is a suitable candidate to calculate and predict weak decay modes across generations of quarks. The asymmetry between matter and antimatter is an important consequence. Testing of the CKM angles through the decays of mesons have verified the Unitary angles through experiments at BaBar. Through an understanding of 3 forms of CP violating procedures, B-mesons have provided a clean measurements of $\alpha$ and $\beta$  verifying the CKM theory and evidence for a significant level of CP violation.
As shown the first evidence for CPV in the neutral Kaon systems and subsequent discoveries in this sector are important verifications of CPV theory and also verify $CPT$ conservation. The current state of CPV in the charm sector has been discussed, showing that current measurements are not accurate enough to verify CPV in this system. 
A criterion determining possible CPV mechanisms for new physical models has been introduced, highlighting the physical consequences of symmetry in nature. In particular the Aspon model and some of its experimental predictions were shown to contradict recent experimental data.
During a cosmological phase transition the magnetic helicity density is related to the cosmic baryon number density and from this the CP violation responsible for the excess of matter over antimatter also provides helicity to the magnetic field.