\section{$CP$ and Conservation}
\vspace{-1.0em}
\begin{center}
\tiny{\textit{Kevin Maguire}}
\end{center}

It has been shown that the violation of $C$ and $P$ are large effects. In fact, they are both maximally violated by the weak force. CP symmetry was proposed to reconcile these two quantities. This new symmtery was of course conserved by the strong and EM forces and seemed to be conserved in the weak force. There is good evidence to suggest that $CP$ is conserved in weak decays of leptons and the helicity of neutrinos.

In considering decays of polarized muons to electron final states:

\begin{align*}
\mu^{-} \rightarrow e^{-} \bar{\nu}_{e} \nu_{\mu} & & \mu^{+} \rightarrow e^{+} \nu_{e} \bar{\nu}_{\mu} \\ 
\end{align*}

\noindent It is found that the electrons are emitted more frequently in certain directions. Thus there is an asymmetry, $\xi_{\pm}$ in the emission, with the two decays having different asymmetries. Under the transformation $\hat{C}$ it is expected that these two decays would have the same asymmetry. Similarly, by the $\hat{P}$ transformation it is expected that the distributions should be completely uniform and favour no direction. It was clear that $C$ and $P$ were being violated, but it was also noted that $\mu^{+}$ and $\mu^{-}$ have the exact same lifetimes. For this system $CP$ conservation implies that the probability of an electron in one direction should be equal to the probability of a positron being emitted in the opposite direction. Using the experimentally determined formula for muon decay assymetries it is found that $CP$ conservation implies the $\xi_{+} = - \xi_{-}$. Experimentally these values are measured as $\xi_{+} = - \xi_{-} =  - 1.00 \pm 0.04$ \cite{Martin+Shaw}. Thus it is found that $CP$ is conserved in this leptonic system. In fact, there is no experimental evidence for the the violation of $CP$ in weak leptonic decays.

\noindent The action of $\hat{C}$ and $\hat{P}$ on neutrinoes with definite handedness is discussed here. Handedness, also known as helicity is defined as right if the projection of a partilces spin, $m_{l}$, is in the same direction as the particles motion. This corresponds to $m_{l}$ having the same sign as the particles momentum. Similarly a left handed particle has the projection of its spin in the opposite direction to its momentum. The $\hat{C}$ operator changes a particle into its anti-particle and thus a left handed neutrino transforms to a left handed anti-neutrino. The $\hat{P}$ operator reverses a particles momentum and thus changes its handedness, so a left handed neutrino goes to a right handed neutrino. However, in nautre it is found that only left handed neutrinos and right handed anti-neutrinos are observed. $C$ and $P$ predict that decays involving lefthanded neutrinos and right handed anti-neutrinos should behave the same as decays involving right handed neutrinos and left handed anti-neutrinoes. Thus if none of the latter are observed it is clear that these particles are not treated the same by nature, showing that $C$ and $P$ are violated. The solution here is that the combined operation of $\hat{C}\hat{P}$ converts a left handed neutrino to a right handed anti-neutrino, thus we see that $CP$ conservation requires that only these two definite handed neutrinos are observed in nature.               

This review is, of course, not on $CP$ conservation. As stated, $CP$ is conserved in weak leptonic decays, but this is certainly not the case in hadroic or even semi-leptonic decays. $CP$ violation (CPV) was first observed in the mixing of neutral K-mesons by Christenson, Cronin, Fitch and Turlay in 1964 \cite{FirstCPV}. They observed the $CP = -1$ state $K^{0}_L$ decaying to $2$ pions, a state with $CP = 1$. Although the fraction of $K^{0}_L$ decays violating $CP$ in this way is tiny, the discovery was significant.      
